\documentclass[a4paper]{article}

\usepackage{listings}

%% Styling
% Remove paragraph indentation
\setlength\parskip{1em}
\setlength\parindent{0pt}

% Change font styling
\renewcommand{\familydefault}{\sfdefault}

\newcommand{\ctitle}[3]{
    \begin{center}
        \parskip=14pt
        \vspace*{3\parskip}

        {\huge #1}

        {\small #2}

        {\large #3

            \today}

        \rule{7cm}{0.4pt}\\
    \end{center}
}

\newcommand{\clrpage}{\thispagestyle{empty}
    \newpage}

\begin{document}
\ctitle{Git-lathund}{}{}
\clrpage

'\$' avser ett nytt kommando, det vill säga alla nya rader bör köras separat.

\section{Arbetsflöde}
Sektionerna görs i ordning med undantag för sektion~\ref{status}.

\subsection{Se statusuppdatering}
\label{status}
En statusuppdatering av repot fås genom:
\begin{lstlisting}[language=bash]
    $ git status
\end{lstlisting}

Statusuppdateringen visar vilka filer som ligger i index, working directory etc. Referera till bilden på Discord.
\emph{git status} används flitigt för att ta reda på vad som kommer commitas till det lokala repot, eller för att se vilka filer som är versionshanterade.

\subsection{Hämta hem ändringar}
För att hämta hem ändringar från remote (GitHub) används:
\begin{lstlisting}[language=bash]
    $ git pull
\end{lstlisting}
När man börjar arbeta bör alla ändringar hämtas hem.
Detta för att undvika stora konflikter senare och för att alla ska kunna arbeta mot den senaste versionen i den mån det är möjligt.

Varje gång man börjar arbeta bör man inte ha några ändringar på sin lokala dator, därmed fås inga merge-konflikter.
I det fall en merge-konflikt uppstår, se sektion~\ref{merge}.

\subsection{Hantering av mergekonflikt}
\label{merge}
Vid \emph{git pull} uppstår ibland en s.k. mergekonflikt.
En mergekonflikt uppstår när du ändrat något som någon annan också ändrat.
Mergekonflikten löses genom att öppna dokumentet eller dokumenten där konflikten uppstått.
Lättaste sättet att ta reda på vilka filer som är berörda görs med \emph{git status}.
Hur en mergekonflikt löses beror på program och kan skilja sig väldigt mycket.

När konflikten hanterats läggs filerna till med hjälp av \emph{git add}.
Sedan görs en \emph{git commit}.
Texten som autogenererats som commit-meddelande bör inte ändras.

\subsection{Göra ändringar}
Nu kan du börja ändra dina filer med olika program.
Detta kan vara en kodeditor, texteditor eller liknande.
Du kan även lägga till s.k. binary files, det vill säga till exempel bilder, pdf-er eller liknande.

\subsection{Lägga till ändringar i index och commit}
Genom att använda:
\begin{lstlisting}[language=bash]
    $ git add filnamn
\end{lstlisting}
är det möjligt att lägga till filer i index.
Alla ändringar i index kommer att commitas.
Du kan se vilka ändringar som är staged (ligger i index) genom deras gröna färg när du gör en \emph{git status}.

I det fall en version av filen finns i index kan:
\begin{lstlisting}[language=bash]
    $ git add -u
\end{lstlisting}
användas för att lägga till alla uppdaterade (u) filer till index.

För att sedan lägga till alla stagade ändrigar till ditt lokala repo används:
\begin{lstlisting}[language=bash]
    $ git commit -m "meddelande"
\end{lstlisting}
eller
\begin{lstlisting}[language=bash]
    $ git commit
\end{lstlisting}

\subsection{Skicka upp ändringar}
Att skicka alla ändringar från ditt lokala repo till remote (GitHub) görs genom kommandot:
\begin{lstlisting}[language=bash]
    $ git push
\end{lstlisting}
I det fall ett felmedellande kommer upp som säger att remote ligger före din lokala branch bör en \emph{git pull} göras.
Eventuella konflikter bör då hanteras och sedan görs en \emph{git push} igen.

\end{document}